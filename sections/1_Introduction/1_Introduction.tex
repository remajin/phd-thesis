As delegation does not rely on shared memory, it can serve as a powerful programming
model for both concurrent and distributed computing. We present two system, \trust{} and
\name{}, to demonstrate the versatility of this approach across different domains.

\trust{} introduces a type- and memory-safe alternative to traditional locking in concurrent
programs. Instead of synchronizing multi-threaded access to an object of type T with a lock,
it can be encapsulated in a \trust{} with a designated trustee thread handling all operations
through message passing. This delegation-based model eliminates per-lock throughput
limitations, achieving up to 22x higher throughput in microbenchmarks and 5-9x
improvement for a key-value store in high-contention workloads, while remaining
competitive even without contention.

\name{} extends the concept of delegation to distributed computing. It allows an
application written for a single machine to scale to a small cluster with minimal
changes. Using Remote Direct Memory Access (RDMA) for high-performance
communication, \name{} supports delegating Rust closures across machines and
employs lightweight user-space threads (fibers) for concurrency. Our experiments show 3x
better throughput scaling than eRPC and performance comparable to the Graph500 MPI
reference implementation.

Together, \trust{} and \name{} illustrate how delegation unifies concurrency and
distribution under a safe, efficient, and scalable programming paradigm, enabling high-
performance computation without the complexities of shared memory or explicit
synchronization.

The main contributions of this dissertation are summarized below:
\begin{itemize}
	\item \trust{}: a model for efficient, multi-threaded, delegation-based programming with shared 
	objects leveraging the Rust type system.
	\item \name{}: an extension of \trust{} that enables a normal delegation based application to run 
	on and utilize the resources of a rack with minimal changes to application code.
\end{itemize}

The rest of this chapter introduces both of these and then outlines the rest of this dissertation.