\section{Background on RDMA}
RDMA is a networking approach that allows one machine to directly access a remote machine's memory. It uses 
the user-level zero-copy transfers to minimize the involvement of remote machine's operating system and CPU 
as opposed to traditional TCP/IP stack that involve both on each machine heavily. There are many 
implementations of this concept \cite{946704,pfister2001introduction,5289144,87568}, most popular of which 
are InfiniBand, RoCE (RDMA over Converged Ethernet), and iWARP (internet Wide Area RDMA Protocol). The 
results presented in this paper are obtained by using RoCE.

\subsection{RDMA API}
\label{s:rdmaapi} 
RDMA hosts communicate using queue pairs (QPs) that consist of a send queue and a receive queue, and are 
maintained by the NIC. Applications post operations to these QPs by using functions called \textit{verbs}. 
For remote access the remote machine first needs to register a memory region with the NIC. The NIC driver 
pins this region in physical memory. The address and a key related to this region then needs to be exchanged 
between the machines out of band (i.e. without using RDMA). After this exchange, the remote memory can be 
accessed without involving either of remote operating system or CPU. This is called 
\textit{RDMA Memory Semantics}, and uses \textit{verbs} \textit{READ} and \textit{WRITE}.

RDMA also provides \textit{Messaging Semantics} that use \textit{verbs} \textit{SEND} and \textit{RECV}. In 
this case receiver has to post a \textit{RECV} \textit{verb} before the sender can send the data. In this 
regard it is similar to an unbuffered sockets implementation. Just like \textit{Memory Semantics}, 
\textit{Messaging Semantics} also bypass the remote kernel but unlike \textit{Memory Semantics}, it has to 
involve remote CPU to post a \textit{RECV}. These \textit{verbs} also have slightly lower latency than 
\textit{READ}  and \textit{WRITE} \cite{180191,179767}.

\subsection{Transport types}
RDMA transports are either connected or unconnected (also called datagram), and either reliable or unreliable. 
Connected transport require a one-to-one connection between two QPs. If an application wants to communicate 
with \textit{N} machines, it will need to create \textit{N} QPs. With unconnected transport one QP can 
communicate with many QPs. For reliable transport NIC uses acknowledgments to guarantee in-order delivery and 
return an error code on failure, while unreliable transport does not provide any such guarantee. InfiniBand 
and RoCE use lossless link layer, so even in case of unreliable transports, losses are pretty rare and happen 
because of bit error or link failure. In case of connected transport a failure will break the connection. Not 
all transports provide all of the verbs. Table \ref{tab:rdma} gives an overview of transport types and the 
verbs they support. Current implementations of RDMA only provide reliable connected (RC), unreliable 
connected (UC) and unreliable datagram.

\begin{table}[t]
  \centering
  % Use .5\textwidth to only have the table fill half the page,
  % and \textwidth for the full page
  \resizebox{.4\textwidth}{!}{
	\begin{tabular}{llll}
		Verb       & RC       & UC       & UD \\ \hline
		SEND/RECV  & \cmark   & \cmark   & \cmark \\
		WRITE      &  \cmark  &  \cmark  &  \xmark  \\
		READ       &  \cmark  &  \xmark  &  \xmark
	\end{tabular}
  }
	\caption{Operations supported by each transport type.
	\label{tab:rdma}}
\end{table}