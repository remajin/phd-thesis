\section{Future Work}

Below we discuss some avenues for further research and improvements to \trust{} and \name{}.

\subsection{Better Profiling Tools}
Due to the polling nature of trustees, it can be difficult to gauge how much useful work any given trustee is 
doing at a particular point in the application's run time. This mismatch can cause expected performance to differ 
from observed performance. As an example, even if some of the trustees are not getting any work, they will 
still show up as busy on profiling tools like htop and perf. Tools that can distinguish between idle spinning and 
productive work would greatly aid in diagnosing performance issues.

\subsection{Better Load Balancing}
Both \trust{} and \name{} leave the responsibility for distributing data across trustees to the application 
developer. While this approach is workable for a proof-of-concept, it complicates application design: any change 
in workload distribution may require substantial data redistribution. This static assignment can also leave 
performance unused—for example, when a few trustees are heavily loaded while others remain mostly idle. A dynamic 
mechanism that detects imbalance at runtime and migrates entrusted data among trustees could potentially improve 
throughput and responsiveness. However, such a solution introduces diagnostic and migration overhead, which may 
be unacceptable when most or all trustees are already heavily loaded. Determining whether dynamic redistribution 
is beneficial, and under what conditions, requires further empirical investigation.

\subsection{Handling Network Conditions}
\name{} relies on ideal network conditions—for example, environments with no congestion, symmetric routing, and 
no dropped packets. However, such perfectly reliable networks are rare in real-world deployments, which could 
significantly hinder the practical adoption of the protocol. To improve its robustness, additional mechanisms 
for managing adverse network conditions need to be explored. Potential strategies include implementing 
rate-limiting techniques to prevent overload, incorporating retransmission methods to recover from lost packets, 
and leveraging Explicit Congestion Notification (ECN) to proactively signal and respond to network congestion. 
Investigating these approaches presents valuable directions for future research and development.