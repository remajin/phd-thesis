\section{Gossamer Design}
\begin{figure}[ht]
  \centering
  % Use .5\textwidth to only have the table fill half the page,
  % and \textwidth for the full page
  \includegraphics[width=1\textwidth]{figures/gossamer_design.pdf}
  \caption{High-level design of \name{}}
  \label{fig:design}
\end{figure}


\name{} allows developers to write rack-wide applications just like single machine applications. Programmers will 
use the delegation framework to write applications that are distributed by \name{} with very few changes to the 
code. \ref{fig:design} shows a high level view of what a rack with multiple \name{} applications looks like. 
Developers use the controller machine to interact with the rack and launch applications. Here controller machine 
is just a normal machine running \name{} processes like any other machine in the rack, the only significance is 
that this is the machine that users log into for terminal access. Each machine in the rack has a daemon running 
that is always listening for instructions from the controller machine. The daemon is responsible for determining 
the topology of the rack and managing the applications. A machine can have multiple applications running on it at 
the same time. Rack applications that are running simultaneously are isolated from each other like multiple 
processes on a single machine. The rest of this section describes some design details that are particularly 
interesting.

\subsection{Gossamer Daemon}
\begin{figure}[ht]
  \centering
  % Use .5\textwidth to only have the table fill half the page,
  % and \textwidth for the full page
  \includegraphics[width=.8\textwidth]{figures/process.png}
  \caption{A Gossamer process on one machine}
  \label{fig:process}
\end{figure}


As shown in \ref{fig:design}, each machine has \name{} daemon running. These daemons connect with each other to 
discover the rack size and store metadata like gossamer-id, machine-ip etc. The daemon is responsible for 
managing any \name{} processes running on the rack at any time. To launch a \name{} application, users need to 
run the application binary on the controller machine only. The application will contact the local daemon and tell 
it to start the process specifying how many system threads the whole \name{} process should use. This daemon will 
contact the rest of the rack and tell the daemons there to launch the process as children. The daemon is also 
responsible for detecting crashes on any local instance of a \name{} process and terminating it across the whole 
rack. This means that \name{} processes are fate sharing in the same way single machine processes crash if a 
single thread crashes.

\subsection{Gossamer Process}
To launch a \name{} process, the user logs into the controller machine and starts the application there. Since 
parts of \name{} rely on similar memory layout (as discussed in Section \ref{sg:slots}), each machine in the rack 
should be populated with the same binary ahead of time. The application needs to have its \textit{main} 
function call a helper function from the \name{} library and pass a function that would act as the real 
\textit{main} function along with how many threads the application needs to use across the whole rack. This 
function will be referred to as \textit{gsm\_main} for the rest of the chapter. The process consists of many 
fibers per system thread for concurrent operation as shown in \ref{fig:process}. Fibers are lightweight, 
user space threads that can spawn on any system thread that is part of the \name{} process, and on any machine 
in the rack. Fibers that need to access shared data make delegation requests consisting of closures that modify 
the data as needed. From the point of view of the operating system, a single instance of \name{} process behaves 
like any other normal process. The main difference is that instead of using system calls like 
\textit{getpid()}, a \name{} process contacts the daemon to get any metadata needed.
\begin{table}

\begin{subtable}[t]{1\textwidth}
\centering
\begin{lstlisting}
fn main() {
    let pool = Threadpool::new(4);
    let property = HashMap::new();
    let trust = pool.entrust(property);
    pool.spawn(|| {
        trust.apply(|property| property.insert(``Greeting'',``Hello''));
    });
}
\end{lstlisting}
\caption{A small example of delegation code with \trust{}}
\label{tab:delegationexample1}
\end{subtable}
		
\begin{subtable}[t]{1\textwidth}
\centering
\begin{lstlisting}
fn main() {
    gossamer::main(gsm_main, 4);
}

fn gsm_main() {
    let property = HashMap::new();
    let trust = gossamer::entrust(property);
    gossamer::spawn(|| {
        trust.apply(|property| property.insert(``Greeting'',``Hello''));
    });
}
\end{lstlisting}
\caption{Same delegation request with \name{}}
\label{tab:delegationexample2}
\end{subtable}

\caption{\trust{} vs \name{} code differences}
\label{tab:trust_vs_gsm}
\end{table}

\ref{tab:trust_vs_gsm} shows a minimal \trust{} example and how the same program looks like when using \name{}.

\subsection{Trustee, Trust and Property}
\input{figures/trustntrustee}

\name{} builds on the Trust/Trustee concept from \cite{ahmad2024trust} (Chapter \ref{trust} in this dissertation), 
which we introduce briefly here. A \textit{Trustee} is a worker fiber that processes delegation requests and sends 
back the response. The application can entrust some data to a trustee if the delegated work needs access to it and 
receive a \textit{Trust} that holds relevant metadata. This entrusted data is referred to as \textit{Property}. 
The Trust can be used to delegate any work that needs access to this property. \ref{fig:trustntrustee} shows the 
relationship between Trust, Trustee and Property. A Trust can be cloned and shared with other fibers so that 
they can also start delegating work that needs access to same property. Multiple properties can be entrusted 
to a single trustee at the same time. Trustees can act as remote or local trustees based on where the fiber 
trying to access the property lives in the rack.

% \subsection{Delegation Workflow}
% \label{s:del_flow}
% \begin{table}
\begin{lstlisting}
let property = HashMap::new();
let trust = gossamer::entrust(property);
gossamer::spawn(|| {
    trust.apply(|property| property.insert(``First Greeting'',``Hello''));
    trust.apply_then(|property| property.insert(``Second Greeting'',``Hi''), 
        |_| println!(``added second greeting''));
});
\end{lstlisting}
\caption{A small example of delegation code}
\label{tab:delegationexample}
\end{table}

% \name{} provides both blocking and nonblocking/asynchronous delegation operations. In case of blocking operations, the 
% delegating fiber will wait for the response from the trustee before continuing. For nonblocking delegation, the 
% delegating fiber will instead provide a callback closure to be executed upon completion of the request, and 
% continue without waiting. First we will describe the workflow for blocking delegation requests and then specify 
% how that differs from the nonblocking delegation. Each fiber issues a request that is put in a pending queue 
% before voluntarily yielding the runtime. Each thread has a fiber that periodically checks for any incoming 
% responses and sends any requests in the pending queue. As there will be many fibers running on each thread this 
% will allow the polling fiber to send multiple requests as a larger batch, increasing the throughput of the system. 
% The requests originating from a client that are destined for the same trustee are batched in a 
% \textit{RequestSlot} (discussed in detail in Section \ref{sg:slots}) that can hold a variable  number of requests depending 
% on their size. On the trustee side, one of the fibers is dedicated to polling for any incoming requests. Since any 
% incoming requests in a RequestSlot will be contiguous in memory, it can complete all of them and then send all of 
% the responses in a single batch as well. Similar to requests, responses are batch using a \textit{ResponseSlot}. 
% The fiber that polls for responses on the requesting thread will then process the responses and add the 
% corresponding fibers back in to the ready to run queue. The main difference for async delegation is that instead 
% of yielding after enqueuing the request, it just keeps going and enqueues any subsequent requests as well. This 
% fiber will yield the runtime when the pending queue is full, after which the response polling fiber can run and 
% send all of the requests to the respective trustees. Once it receives the responses, it can execute any callback 
% closures the user might have provided. This means that there is no benefit to having multiple fibers per thread 
% that issue requests as that is not needed for batching and the fibers will be yielding far less frequently. 
% \ref{tab:delegationexample} shows a small example program that uses both types of delegation.

\subsection{NoRef}
\label{sg:noref}

\name{} leverages the Rust type system to prevent the sending of references over delegation channels. One of 
the first challenges that arise when extending delegation to multiple machines is the fact that any pointers 
or references captured by a delegated closure will not be valid on a remote machine. This reference or pointer 
will result in undefined behavior on any machine other than where it originated. To prevent this, \name{} uses 
a combination of rust features named \textit{auto\_traits} and \textit{negative\_impls}. In rust, traits define 
the behavior of data types. They inform the compiler what functionality a type has and can be used to restrict 
which data types can be passed to a generic function. Auto\_traits is a feature that automatically implements 
a trait for every data type, be it an existing type or user defined. A negative implementation is used to 
exclude a data type from the auto implementation. \name{} defines an auto trait called \textit{NoRef}. Then 
all the types that contain a reference or raw pointer are given negative implementation. The function 
\textit{apply}, that is used to send delegation messages as described in previous section, requires all the 
closures to comply with the NoRef trait as shown in \ref{fig:trustntrustee}. This however limits severely what 
type of data can be captured by any delegated closure as many frequently used types like strings and vectors 
contain pointers internally. To get around that \name{} provides a way to send any data type that can be 
serialized along with the closure as an extra parameter to the delegated closure. If the programmer makes a 
mistake and tries to use a closure that has captured a reference for delegation, a custom error message at 
compile time will inform them what the issue is and direct them to use the serialization method instead.

\subsection{Request size and TCP fallback}
\label{s:tcp_fallback}
The requests going from a client thread to the same trustee are batched and 
on trustee side the requests coming from a single thread are all contiguous in memory. This is achieved by 
the trustee having a pre-allocated space in memory (called request slot) for incoming requests from each client 
thread. This limits how many requests a client thread can send in a single batch depending on how much data is 
being sent with each request. As an example if each request captures 32 bytes of captured data, the 
total request size is 40 bytes. Next section goes into detail about the format of a \name{} delegation request. 
Assuming 1 KB request slot, one batch can have at most 25 requests. While this allows us to optimize for small 
requests it also places a hard limit on how much data a request can capture i.e. the size of the request slot. 
To work around this limit and support larger requests, \name{} uses TCP to send any captured data larger than 
that separately, while the request header goes in the request slot as normal. This degrades the performance 
severely as TCP is much slower than RDMA. We use TCP here instead of RDMA because, as mentioned in Section \ref{s:rdmaapi}, 
RDMA needs the memory used to be pre-registered with the NIC, placing an upper limit on how much data can be sent 
using RDMA at once, i.e. the size of pre-registered buffer. Since there would be an upper limit anyway and larger 
requests are not the focus for \name{}, we decided to opt for the simpler design in place at the moment.

\subsection{Request/Response Slot Format}
\label{sg:slots}

\begin{figure}[ht]
  \centering
  % Use .5\textwidth to only have the table fill half the page,
  % and \textwidth for the full page
  \includegraphics[width=.8\textwidth]{figures/req_slot.pdf}
  \caption{Request and Response slot layout}
  \label{fig:req_slot}
\end{figure}


As discussed in the previous section, the number of requests sent in a batch depends on the size of the requests 
plus the size of any headers sent with the request. Section \ref{st:slots} describes the format of request and 
response slots used for single machine delegation channel. Here we describe further optimizations made to request
slots to minimize per request metadata to reduce wasted network bandwidth. The two-slot optimization of single 
machine delegation channel is also disabled for remote delegation, as the cost of doing two separate RDMA write 
operations outweighs any benefit gained from it. \\

The necessary parts to process a \name{} request are as follows:
\begin{itemize}
	\item The closure. Closures in rust are fat pointers consisting of a vtable that points to where the 
	code is in the text section along with the size of captured data, and a data pointer that points to 
	the captured data.
	\item The pointer to where the property is located on trustee.
	\item Length of serialized data.
	\item The captured data.
	\item Any serialized data.
\end{itemize}

While we can't avoid sending the captured data, serialized data and the property pointer, \name{} employs 
some strategies to minimize the information that needs to be sent inside the request slot. As the data pointer 
within the closure is not going to be valid on a remote machine, there is no need to include that in the request. 
Since the same binary is running on all the machines in the system, they all have access to any information 
known at compile time. This includes all of the information in the vtable, provided the vtable pointer. Here, 
we make the observation that for the vast majority of the runtime of an application, most of the requests in 
the request slot will be about a single closure or a few closures at most, meaning we can have a small lookup
table for a few vtable pointers in the request slot instead of including one with every request. Combining that 
with the start point of data received, the trustee can reconstitute the closure locally, removing the need to 
send the closure in the request slot. If there are indeed more closures in the request slot than will fit in the 
lookup table, any extras can be sent in the request slot. This will reduce the data being sent in the common 
case with minimal overhead, but will be no more expensive in the special case. Similarly, if the serialized data 
has a size known at compile time the trustee already has access to it, so the only time it needs to be sent 
with the request is if it is not known at compile time. One thing to note here is that we rely on same vtable 
pointer to be the valid on all of the machines which is not the case normally due to linux address space 
layout randomization (ASLR), making disabling it a necessity for \name{} to function.\\

Similarly, the responses for all of the requests in a request slot are batched in a response slot and written to 
the client's memory with a single RDMA write. Unlike a request however, the size of a response is not always known 
when submitting a request. If the response size cannot be determined statically, it is preceded by 8 bytes that 
contain its size. While a known response size can be used to limit the number of requests in a request slot just 
like the size of requests, unknown size responses can lead to scenarios where not all of the responses for a 
request slot fit in a single response slot. In such cases the trustee will send anything that doesn't fit in the 
response slot to the client using TCP in a similar fashion as discussed in Section \ref{s:tcp_fallback}. In addition, if 
the response size is zero and statically known, then nothing is sent in the slot for that particular response. 
\ref{fig:req_slot} shows the format for both the request and response slots.\\

In addition to the request/response, the slots also contain some metadata. They both contain a ready bit and
an offset, with the request slot containing a few more things. The ready bit is there to let the other side 
know of a new request or response. It has to be at the end to ensure that when a client or trustee sees the
ready bit the rest of the slot has already been written. This is necessary because RDMA writes data sequentially
and having the start of a slot been written to, does not mean all of it is available to read. The offset has to
do with another optimization for the size of RDMA writes. Depending on the size of requests and responses, a 
slot can be sent to the remote machine before it is 100\% full, in which case writing the full slot to remote 
machine's memory wastes bandwidth and lowers performance. Since the ready bit needs to be at the end of the 
slot and in a known place, requests and responses are aligned to the end of the slot leaving the start of it
empty. The offset informs the remote side where the actual data starts in the slot. The request slot also 
contains the request count and the vtable cache as discussed earlier. The last thing in request slot metadata
is a bit that indicates if there is a single large request in the slot or multiple small ones. As discussed 
in Section \ref{s:tcp_fallback}, if the request will not fit in the slot it is sent using TCP and only the relevant 
pointers go in the slot. The decision to only send a single request if it is large is to simplify the design
since large requests are not the focus of this work.

\subsection{Trustee and Client memory layout}

\begin{figure}[ht]
  \centering
  % Use .5\textwidth to only have the table fill half the page,
  % and \textwidth for the full page
  \includegraphics[width=\textwidth]{figures/req_mem_layout.pdf}
  \caption{Memory layout for Request Slots in client and trustee memory for a rack of three machines where each 
  machine has three system threads.}
  \label{fig:req_mem_layout}
\end{figure}

% \begin{figure}[ht]
  \centering
  % Use .5\textwidth to only have the table fill half the page,
  % and \textwidth for the full page
  \includegraphics[width=\textwidth]{figures/resp_mem_layout.pdf}
  \caption{Memory layout for Response Slots in client and trsutee memory}
  \label{fig:resp_mem_layout}
\end{figure}

% \begin{figure*}[ht]
  \begin{subfigure}[t]{\linewidth}
    \includegraphics[width=\textwidth]{figures/req_mem_layout.pdf}
    \caption{Memory layout for Request Slots in client and trsutee memory}
    \label{fig:req_mem_layout2}
  \end{subfigure}
  ~
  \begin{subfigure}[t]{\linewidth}
    \includegraphics[width=\textwidth]{figures/resp_mem_layout.pdf}
    \caption{Memory layout for Response Slots in client and trsutee memory}
    \label{fig:resp_mem_layout2}
  \end{subfigure}
\end{figure*}

The clients and trustees on a single machine share memory, so a client can prepare the whole request slot in a 
pre-designated place and then flip the ready bit at the end to signal the trustee that it is ready. The trustee 
continuously polls this piece of memory to determine the arrival of new requests. This leads to $num\_threads^2 
\times sizeof(reques\_slot)$ bytes of memory being reserved for request slots. Same for response slots as well.
We can think of that as a \textit{square} in memory with $num\_threads$ side length. When we move on to a 
distributed setting, we can extend this \textit{square} to have $num\_threads \times num\_machines$ side length.
The whole of the bigger \textit{square} does not need to be allocated on each machine as that would waste memory.
Instead, parts of it are allocated on each machine. \ref{fig:req_mem_layout} shows an example layout for a system 
with three machines that have three system threads each for a total of nine threads. In the figure, each cell 
represents a request slot going from a source thread to a destination thread as labeled and horizontal
rows represent memory that is physically contiguous while different colors divide the \textit{square} based on which 
machine the memory is allocated on. Here, unlike in a single machine setting, there are two \textit{squares}. 
Clients on one machine need to prepare the requests in local memory before writing the request slot to the trustee's 
memory on another machine. Theoretically, RDMA allows the clients to write the requests in the trustee's memory as
they come and flip the ready bit once the slot is full similar to single machine channel, however this would result 
in many small RDMA write operations, reducing the effectiveness of batching and sinking the performance. 

From the perspective of the whole rack as a single system, each row of request slots in the client memory 
\textit{square} maps to a column in the trustee memory \textit{square}. For example a request slot going from 
thread 1 (thread 1, machine 1) to thread 7 (thread 1 machine 3) is prepared in memory represented by the cell in
row one and column seven in client memory \textit{square} and then written to the cell in row seven and column one
in trustee memory \textit{square} and so on. \ref{fig:req_mem_layout} also shows a few additional examples. There 
is a similar system in place for response slots as well.

\subsection{Scaling impact of increasing number of Queue Pairs}

\begin{figure}[ht]
  \centering
  % Use .5\textwidth to only have the table fill half the page,
  % and \textwidth for the full page
  \includegraphics[width=.8\textwidth]{figures/plots/qp_scaling_6m_r6615.pdf}
  \caption{Throughput of RDMA writes in MOPs for quadratic and reduced number of QPs. Each write is 256 bytes.}
  \label{fig:qp_scaling_6m}
\end{figure}

As discussed in Section \ref{s:rdmaapi}, RDMA communications happen via a queue pair (QP). Theoretically each machine
pair only needs to have a single QP resulting in $num\_machines^2$ QPs. However, this means sharing a QP among 
many threads. The QPs use a spin lock internally to make sure QPs can be shared safely, but that results in threads
having to wait for each other before they can send requests to a trustee. A better but naive design might be
to have a QP for every pair of system threads in the whole rack resulting in QPs in the order of 
$(num\_machines \times num\_threads\_per\_machine)^2$. This also results in lost performance due to NIC cache 
limitation. The NIC caches any recently used QPs, along with page mappings for any recently used pinned pages
in its SRAM. Increasing the number of QPs results in NIC running out of SRAM, which in turn results in cache 
misses. We use a middle of the road system where instead of each thread making a separate QP for every other 
thread, it makes a QP for each machine in the rack. This reduces the total number of QPs in the system to 
$num\_machines^2 \times num\_threads\_per\_machine$. \ref{fig:qp_scaling_6m} shows an experiment that demonstrates 
the performance gained by reducing the number of QPs in the system. This experiment was performed with 6 machines, 
all connected to switch with 100Gb/s ethernet links. Each thread writes 256 bytes in a randomly chosen remote 
thread's memory twenty million times. As the number of threads increases, thereby increasing the number of QPs, 
the system with reduced number of QPs maintains peak throughput whereas the system with quadratic number of QPs 
starts to fall off, resulting in a 25\% decrease at the extreme end.