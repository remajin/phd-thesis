\section{Background on RDMA}
RDMA is a networking approach that allows one machine to directly access a remote machine's memory. It uses 
the user-level zero-copy transfers to minimize the involvement of remote machine's operating system and CPU 
as opposed to traditional TCP/IP stack that involve both on each machine heavily. There are many 
implementations of this concept \cite{946704,pfister2001introduction,5289144,87568}, most popular of which 
are InfiniBand, RoCE (RDMA over Converged Ethernet), and iWARP (internet Wide Area RDMA Protocol). The 
results presented in this chapter are obtained by using RoCE.

\subsection{RDMA API}
\label{s:rdmaapi} 
RDMA hosts communicate using queue pairs (QPs) that consist of a send queue and a receive queue, and are 
maintained by the NIC. Applications post operations to these QPs by using functions called \textit{verbs}.
Each queue can be associated with a completion queue, that can be polled to learn the status of posted 
operations. The completion queue can be shared between both parts of the QP. 
For remote access the remote machine first needs to register a memory region with the NIC. The NIC driver 
pins this region in physical memory. The address and a key related to this region then needs to be exchanged 
between the machines out of band (i.e. without using RDMA). After this exchange, the remote memory can be 
accessed without involving either of remote operating system or CPU. This is called 
\textit{RDMA Memory Semantics}, and uses \textit{verbs} \textit{READ} and \textit{WRITE}.

RDMA also provides \textit{Messaging Semantics} that use \textit{verbs} \textit{SEND} and \textit{RECV}. In 
this case receiver has to post a \textit{RECV} \textit{verb} before the sender can send the data. In this 
regard it is similar to an unbuffered socket's implementation. Just like \textit{Memory Semantics}, 
\textit{Messaging Semantics} also bypass the remote kernel but unlike \textit{Memory Semantics}, it has to 
involve remote CPU to post a \textit{RECV}. These \textit{verbs} also have slightly lower latency than 
\textit{READ}  and \textit{WRITE} \cite{180191,179767}.

\subsection{Transport types}
RDMA transports are either connected or unconnected (also called datagram), and either reliable or unreliable. 
Connected transport require a one-to-one connection between two QPs. If an application wants to communicate 
with \textit{N} machines, it will need to create \textit{N} QPs. With unconnected transport one QP can 
communicate with many QPs. For reliable transport NIC uses acknowledgments to guarantee in-order delivery and 
return an error code on failure, while unreliable transport does not provide any such guarantee. InfiniBand 
and RoCE use lossless link layer, so even in case of unreliable transports, losses are pretty rare and happen 
because of bit error or link failure. In case of connected transport a failure will break the connection. Not 
all transports provide all of the verbs. \ref{tab:rdma} gives an overview of transport types and the 
verbs they support. Current implementations of RDMA only provide reliable connected (RC), unreliable 
connected (UC) and unreliable datagram.

The meaning of a success status while polling a completion queue depends on the type of transport the related
send/receive queues are using. A work completion is generated for all receive requests and added to the 
completion queue. For send requests the work completion is only generated if the signaled bit was set to true
in the work request or the send request ended with an error. In case of reliable transport, a success for
send requests indicates that the data has been written to the remote machine's memory. For unreliable 
transport however, a success only means that the local buffers can be reused safely, with no indication of 
the remote memory's status. A successful work completion also indicates that any unsignaled work posted 
earlier has finished successfully \cite{pollcq}. \name{} uses RC QPs to ensure any data loss in the network is caught.
We also use the cumulative nature of work completion successes to our advantage by not using signaled send 
for every RDMA write and only having a signaled send after 1000 unsignaled sends. This allows us to cut down 
on time spent while polling the completion queue.

\begin{table}[t]
  \centering
  % Use .5\textwidth to only have the table fill half the page,
  % and \textwidth for the full page
  \resizebox{.4\textwidth}{!}{
	\begin{tabular}{llll}
		Verb       & RC       & UC       & UD \\ \hline
		SEND/RECV  & \cmark   & \cmark   & \cmark \\
		WRITE      &  \cmark  &  \cmark  &  \xmark  \\
		READ       &  \cmark  &  \xmark  &  \xmark
	\end{tabular}
  }
	\caption{Operations supported by each transport type.
	\label{tab:rdma}}
\end{table}