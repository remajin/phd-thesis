\section{Evaluation}
\label{sg:eval}

In this section we use microbenchmarks to evaluate \name{} on throughput and latency. We also use two application
benchmarks to evaluate how \name{} performs in the real world. These experiments were performed on the r6615 
cluster from Cloudlab Clemson using 12 machines. Each machine here uses AMD EPYC 9354P CPU and has 32 threads 
over four NUMA nodes at 3.3 GHz. We do not use hyperthreading for this experiment. Each machine has 100 Gbps 
Mellanox ConnectX-6 interfaces and connected to a Dell Z9432 switch. These machines are equipped with 188 GB RAM 
each and are running Ubuntu 22.04.2 LTS (GNU/Linux 5.15.0-160-generic x86\_64).


\subsection{Microbenchmark: Throughput}
\begin{figure}[ht]
  \centering
  % Use .5\textwidth to only have the table fill half the page,
  % and \textwidth for the full page
  \includegraphics[width=.8\textwidth]{figures/plots/echo_tput_r6615.pdf}
  \caption{throughput vs number of threads for eRPC vs mpi vs gossamer.}
  \label{fig:echo_tput}
\end{figure}


We use an echo application to evaluate the throughput scaling of \name{} as the number of machines in the rack 
increases. For this experiment, each trustee/server sends the same data it received as a response. Each thread 
acts as both client and server. Clients choose a server at random form a uniform distribution and send a request 
containing 32 bytes of data. Each client sends ten million asynchronous echo requests. We compare against eRPC 
\cite{225980} and an MPI library used and developed by graph500 reference benchmark \cite{g500}.

\ref{fig:echo_tput} shows the results of our experiment. Here we use 16 threads per machine as the eRPC benchmark
only supports up to two NUMA nodes. Each successive data point has one more machine than the previous one, resulting 
in an increment of 16 threads. eRPC starts with 2.9 MOPs per thread and maintains that as more threads are
added. This is in line with their reported performance in a github issue \cite{erpc_issue} for two machines 
connected with a 100 Gbps connection. \name{} starts with double the throughput and scales 3x better than eRPC.
Similarly, though the MPI application starts out performing better than \name{}, it falls behind quickly as the 
number of machines increases. 
% This is because MPI is not well suited to continously sending asynchronous requests 
% back and forth with no other work being done in between.

\begin{figure}[ht]
  \centering
  % Use .5\textwidth to only have the table fill half the page,
  % and \textwidth for the full page
  \includegraphics[width=.8\textwidth]{figures/plots/echo_tput_reqsize_r6615.pdf}
  \caption{Throughput across a range of request sizes for eRPC, MPI, and \name{} in an Echo application. This 
  experiment uses 192 threads across 12 machines.}
  \label{fig:echo_reqsize}
\end{figure}

\ref{fig:echo_reqsize} shows the performance of all three systems across a range of request sizes. The MPI system 
performs better than both \name{} and eRPC due to smaller request headers, however as the request size 
increases, this advantage becomes negligible and the other systems overtake it. All three systems experience a 
drop in the number of requests served as the request size increases, however \name{} experiences a particularly 
steep drop as it cross th 1KB threshold. This is explained by the TCP fallback for large requests as discussed 
in section \ref{s:tcp_fallback}. 

\subsection{Microbenchmark: Latency}
\begin{figure}[ht]
  \centering
  % Use .5\textwidth to only have the table fill half the page,
  % and \textwidth for the full page
  \includegraphics[width=.8\textwidth]{figures/plots/erpc_vs_gsm_loaded_latency.pdf}
  \caption{Throughput vs latency under load for \name{} and eRPC in an Echo application with two machines and 
  16 threads per machine.}
  \label{fig:erpc_gsm1}
\end{figure}


Next we evaluate 99.9th percentile latency under load for eRPC and \name{}. We exclude the mpi application from 
this experiment as the mpi library we use is highly optimized for throughput at the cost of latency, resulting in 
it not being competitive. This experiment also uses the echo application with a request size of 32 bytes. As both 
\name{} and eRPC use batching for asynchronous requests, load here is represented by the number of outstanding 
requests at any given time.

\ref{fig:erpc_gsm1} shows the results of this experiment. Each data point here represents a doubling of batch 
size until 32 and then an increment of 16. Both system have increased throughput and latency as the number of 
outstanding requests increases, however eRPC does better at smaller loads in terms of both latency and throughput. 
This is because of the optimizations discussed in \ref{sg:slots}, that result in better performance at higher 
load at the cost of small performance penalty at lower loads, along with the overhead of switching fibers being 
significant at lower load. eRPC then quickly reaches its peak throughput and any increase in load only increases 
the latency whereas \name{} is able to achieve much higher throughput while remaining below 100 \textmu s tail 
latency.

\subsection{Application: Single Source Shortest Path (SSSP)}
\begin{figure}[ht]
  \centering
  % Use .5\textwidth to only have the table fill half the page,
  % and \textwidth for the full page
  \includegraphics[width=.8\textwidth]{figures/plots/sssp_threads_n27_r6615.pdf}
  \caption{Time taken for SSSP on a graph with $2^{27}$ nodes and 32 average degree.}
  \label{fig:sssp_threads_n27}
\end{figure}

\begin{figure}[ht]
  \centering
  % Use .5\textwidth to only have the table fill half the page,
  % and \textwidth for the full page
  \includegraphics[width=.8\textwidth]{figures/plots/sssp_size_r6615.pdf}
  \caption{Time taken for SSSP on graphs with $2^n$ nodes and 32 average degree.}
  \label{fig:sssp_size}
\end{figure}


To understand the real world performance of \name{}, we implement a distributed version of Dijkstra's algorithm 
for shortest path to each node in a weighted undirected graph given a source node. We compare this application to 
the graph500 reference benchmark \cite{g500} for supercomputers. Even though the graph500 benchmark uses the 
delta-stepping algorithm, we observed that both systems traverse roughly equal number of edges, making any 
algorithmic differences negligible. The graphs for these experiments were generated by graph500's implementation 
of kronecker generator with an edge factor of 32. For each experiment we ran sssp on eight randomly chosen source 
nodes and report the average time taken per run.

\ref{fig:sssp_threads_n27} shows the results of SSSP on a graph with $2^{27}$ nodes and average degree of 32, 
as the size of the cluster increases. Each successive data point increments the thread count by 32. Despite 
graph500 being a highly specialized application designed to benchmark supercomputers, \name{} is able to keep up
with it. Similarly, \ref{fig:sssp_size} shows \name{} achieving comparable performance to graph500 across a range 
of graph sizes with 384 threads across 12 machines.


% \begin{figure}[ht]
  \centering
  % Use .5\textwidth to only have the table fill half the page,
  % and \textwidth for the full page
  \includegraphics[width=.8\textwidth]{figures/erpc_vs_gsm_loaded_latency_t56.pdf}
  \caption{throughput vs latency under load for gsm and eRPC with two machines and 28 threads per machine.}
  \label{fig:erpc_gsm2}
\end{figure}

% \begin{figure}[ht]
  \centering
  % Use .5\textwidth to only have the table fill half the page,
  % and \textwidth for the full page
  \includegraphics[width=.8\textwidth]{figures/plots/sssp_threads_n26_r6615.pdf}
  \caption{Time taken for SSSP on a graph with $2^{26}$ nodes and 32 average degree.}
  \label{fig:sssp_threads_n26}
\end{figure}

\begin{figure}[ht]
  \centering
  % Use .5\textwidth to only have the table fill half the page,
  % and \textwidth for the full page
  \includegraphics[width=.8\textwidth]{figures/plots/kv_r6615_ycsbc.pdf}
  \caption{Throughput of key-value stores on up to 8 server clusters and ycsbc workload.}
  \label{fig:kv_ycsbc}
\end{figure}

\begin{figure}[ht]
  \centering
  % Use .5\textwidth to only have the table fill half the page,
  % and \textwidth for the full page
  \includegraphics[width=.8\textwidth]{figures/plots/kv_r6615_ycsbd.pdf}
  \caption{Throughput of key-value stores on up to 8 server clusters and ycsbd workload.}
  \label{fig:kv_ycsbd}
\end{figure}

% \input{figures/qp_scaling_5m_r6615}
% \begin{figure}[ht]
  \centering
  % Use .5\textwidth to only have the table fill half the page,
  % and \textwidth for the full page
  \includegraphics[width=.8\textwidth]{figures/plots/qp_scaling_6m_r6615.pdf}
  \caption{Throughput of RDMA writes in MOPs for quadratic and reduced number of QPs. Each write is 256 bytes.}
  \label{fig:qp_scaling_6m}
\end{figure}

% As \ref{fig:erpc_batch} shows, they had a great logo too.  And they wrote amazing
% albums, like \gls{rtr} and \gls{eofc}, and a couple more.  They even did
% three (!) jingles for Steel Reserve\footnote{The gross malt liquor.}