\section{Distributed Systems}

Just as multithreaded applications exploit multiple cores within a single machine, distributed systems exploit 
resources across multiple machines. To support this model, a variety of distributed systems frameworks have 
emerged, each providing abstractions and tooling to handle communication, coordination, and fault tolerance. Below 
we discuss a few of these briefly.

Remote Procedure Call (RPC) is a widely used model for communication in datacenters
\cite{Birrell:1984:IRP:2080.357392,Srinivasan:1995:RRP:RFC1831,Bershad:1989:LRP:74850.74861,6702617,Adamson2016,Talpey2010,4228186,brabson2011method,shyam2015managing,merrick2015xml,Soumagne2013MercuryER}.
RPCs provide the useful abstraction of running a function on a remote machine, while hiding the underlying network 
communication. For most approaches to RPC, the function needs to be registered at the start and calls to it can 
be either synchronous or asynchronous. A synchronous call will wait for a response while an asynchronous call can 
provide a callback function to be executed once the response is ready and continue execution. This is similar to 
delegation, rather than moving data to where it is needed, the code that uses the data is sent to its location. 
This process typically involves serializing and deserializing of data and transport protocols that ensure reliable 
message delivery. By decoupling distributed computation from explicit message handling, RPC simplifies the 
development of networked applications, supports modular system design, and serves as a foundational mechanism in 
many modern distributed architectures, including microservices and cloud-based systems.

Message Passing is another dominant communication framework
\cite{doi:10.1177/1094342005054257,Doolan:2008:MMP:1497185.1497251,10.1007/978-3-540-30218-6_22,1303192,6154915}. 
Unlike RPC, Message Passing has explicit communication and involves sending and receiving messages. Depending on 
the implementation, there can be guarantees regarding the reliability of message transport or the order of 
messages, or it can be a send-and-forget system where the sender might not even know if the message reached its 
destination. MPI \cite{10.5555/898758} a widely used standard that provides a range of communication primitives, 
from one-to-one messages to many-to-many. It also provides both blocking and non blocking versions of send/receive.

There is also ongoing research to minimize resource usage of communication as much as possible. Remote Direct 
Memory Access (RDMA) is one such attempt \cite{946704,pfister2001introduction,5289144,87568}. RDMA allows the user 
to read a remote machine's memory by utilizing specialized Network Interface Cards (NIC) while bypassing the 
remote machine's operating system. Additionally, there is also some work on using a custom networking stack for 
distributed systems as compared to a hardware accelerated RDMA system \cite{225980,10.1145/3341301.3359657,SocketsvsRDMA}.