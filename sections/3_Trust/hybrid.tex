\section{Hybrid Delegation}

\begin{figure*}[]
    \centering
    \begin{subfigure}[]{0.8\textwidth}
        \centering
        \includegraphics[width=\columnwidth]{figures/plots/motivation-zipfian-95r.pdf}
        \caption{Zipfian}
        \label{hybrid:fig:motivation-zipfian-95r}
    \end{subfigure}
    \centering
    \begin{subfigure}[]{0.8\textwidth}
        \centering
        \includegraphics[width=\columnwidth]{figures/plots/motivation-uniform-95r.pdf}
        \caption{Uniform random}
        \label{hybrid:fig:motivation-uniform-95r}
    \end{subfigure}
    \caption{The throughput of key-value stores using compared systems with different key space size (5\% write)}
    \label{hybrid:fig:motivation-95r}
\end{figure*}

\ref{f:kvstore_tput} shows that delegation's performance advantage depends on the workload and lock contention 
levels. This raises the question of whether it is possible to get the best of both worlds. In his PhD thesis, 
Dr. Chen \cite{chen_thesis} introduced HybridLock, a synchronization mechanism that dynamically switches between 
traditional locking and delegation based on workload conditions. When a HybridLock is created, it assumes low 
contention and starts with locking as default. If, during the runtime, it detects that the lock is highly contented, 
it switches to using delegation manage access to the associated property. By efficiently detecting contention and 
switching between the two modes, HybridLock can leverage delegation to achieve high performance in a highly 
contented system while keeping the overhead during low contention workload to a minimum. \ref{hybrid:fig:motivation-95r} 
is taken from Dr. Chen's thesis and shows that HybridLock can maintain high performance across contention levels.