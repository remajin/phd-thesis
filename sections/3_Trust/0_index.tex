\chapter{\trust{}}
\label{trust}

\trust{} is a general, type- and memory-safe alternative to locking in concurrent programs. Instead of 
synchronizing multi-threaded access to an object of type T with a lock, the programmer may place the object in 
a \trust{}. The object is then no longer directly accessible. Instead, a designated thread, the object's {\it 
trustee}, is responsible for applying any requested operations to the object, as requested via the \trust{} API.
 
While locking offers a limited throughput {\it per lock}, \trust{} is based on delegation, a 
message-passing technique which does not suffer this per-lock limitation. Instead, per-object throughput is 
limited by the capacity of the object's trustee, which is typically considerably higher.
 %giving delegation a theoretical advantage when performance is constrained by a limited number of locks.
 %On the other hand, the latency and overhead of delegation can be higher than that of locking, especially for uncontended locks. 
 
Our evaluation shows \trust{} consistently and considerably outperforming locking where lock contention exists, 
with up to 22$\times$ higher throughput in microbenchmarks, and 5--9 $\times$ for a home grown key-value store, 
in situations with high lock contention. Moreover, \trust{} is competitive with locks even in the absence of lock 
contention. 

% \setcounter{table}{0}
\section{Introduction}
Safe access to shared objects is fundamental to many multi-threaded programs. Conventionally, this is achieved 
through {\it locking}, or in some cases through carefully designed lock-free data structures, both of which are 
implemented using atomic compare-and-swap (CAS) operations. By their nature, atomic instructions do not 
{\it scale} well: atomic instructions must not be reordered with other instructions, often starving part of 
today's highly parallel CPU pipelines of work until the instruction has retired. This effect is exacerbated 
when multiple cores are accessing the same object, resulting in the combined effect of frequent cache misses 
and cores waiting for each other to release the cache line in question, while the atomic instructions prevent 
them from doing other work. These effects are further strengthened in NUMA architectures \cite{203195,10.1145/3591195.3595276,Velten_2022,PAN2021102188}.

Delegation \cite{dice2011flath,calciu2013delegation,petrovic2015delegation,fatourou2011combining,hendler2010flat,oyama1999combining,yew1987combining,shalev2006combining,david2013everything,ffwd}, 
also known as light-weight remote procedure calls (LRPC), offers a highly scalable 
alternative to locking. Here, each shared object\footnote{Here, we use {\it object} to mean a data structure 
that would be protected by a single lock.} is placed in the care of a single core. Other cores issue requests to 
this designated core, specifying operations to be performed on the object.

Compared to locking, where threads typically contend for access, and may even suspend execution to wait for 
access, delegation requests from different clients are submitted to the trustee in parallel and without contention. 
This dramatically reduces the cost of coordination for congested objects. The operations/critical sections are 
applied sequentially in both designs: by each thread using locks, or by the trustee using delegation; here 
delegation may benefit from improved locality at the trustee. Together, this translates to much higher maximum 
per-object throughput with delegation vs. locking.

However, under medium or low contention, classical delegation struggles to compete with locking: 
the latency and overhead of request transmission, request processing and response transmission are insignificant 
compared to the cost of contending for a lock, but can be substantial compared to the cost of acquiring an 
{\it uncontended} lock. 
% \input{sections/3_Trust/back}
\section{\trust{}: The Basics}
\label{st:design}
\begin{table*}[ht]
  \begin{lstlisting}  
let ct = local_trustee().entrust( 17 );           // ct: Trust<i32>
ct.apply( |c| *c+=1 );                             //  c: &mut i32 
assert!(ct.apply( |c| *c ) == 18 );
\end{lstlisting}
  \caption{Minimal \trust{} example. An entrusted counter, referenced by {\tt ct} is initialized to 17, then incremented once. 
  The comments on the right indicate the types of the variables.}
  \label{tab:minimal}
\end{table*}

The objective of \trust{} is to provide an intuitive API for safe, efficient access to shared objects. Naturally, our design motivation is to support delegation, but the \trust{} API can in principle also be implemented using locking, or a combination of locking and delegation. 
Below, we first introduce the basic \trust{} programming model, as well as the key terms {\it trust, property, trustee} and {\it fiber} in the \trust{} context, before digging deeper into the design of \trust{}. 


\subsection{Trust: a reference to an object}


A \trust{} is a thread-safe reference counting smart-pointer, similar to Rust's {\it Arc$<$T$>$}. 
To create a \trust{}, we clone an existing \trust{} or $entrust$ a new object, or $property$ of type $T$, that is meant to be shared between threads. 
Once entrusted, the property can only be accessed by {\it applying} closures to it, using a trust.
Table~\ref{tab:minimal} illustrates this through a minimal Rust example.
Line 1 entrusts an integer, initialized to 17, to the local trustee - the trustee fiber running on the current kernel thread. 
Line 2 applies an anonymous closure to the counter, via the trust.
The closure expected by {\tt apply} takes a mutable reference to the property as argument, allowing it unrestricted access to the property, in this case, our integer. 
The example closure increments the value of the integer. 
The assertion on line 3 is illustrative only. Here, we apply a second closure to retrieve the value of the
entrusted integer\footnote{A note on ownership: While the passed-in closure takes only a reference to the property, the Rust syntax $*c$ denotes an explicit dereference, essentially returning a copy of the property to the caller. This will pass compile-time type-checking only for types that implement $Copy$, such as integers.}.

\subsection{Trustee - a thread in charge of entrusted properties}

In our examples above, \trust{} is implemented using delegation. Here, a {\it property} is {\it entrusted} to a {\it trustee}, a designated thread which executes applied closures on behalf of other threads. 
In the default \trust{} runtime environment, every OS thread in use already has a trustee user-thread ({\it fiber}) that shares the thread with other fibers. When a fiber applies a closure to a trust, this is sent to the corresponding trustee as a message. Upon receipt, the trustee executes the closure on the property, and responds, including any closure return value. This may sound complex, yet the produced executable code %, the amount of work performed is similar to past delegation systems presented by XXXX \cite{xxx} and Roghanchi et. al. \cite{ffwd}, and 
substantially outperforms locking in congested settings.

A TrusteeReference API is also provided. Here, the most important function is {\tt entrust()}, which takes a property of type {\tt T} as argument (by value), and returns a \trust{} referencing the property that is now owned by the trustee. This API allows the programmer to manually manage the allocation of properties to trustees, for performance tuning or other purposes. Alternatively, a basic thread pool is provided to manage distribution of fibers and variables across trustees.% (see \S\ref{s:threadpool}). \na{Threadpool should be elaborated on a bit more somewhere.}


\subsection{Fiber - a delegation-aware, light-weight user thread}

While the \trust{} abstraction has some utility in isolation, it is most valuable when combined with an efficient message-passing implementation and a user-threading runtime. 
User-level threads, also known as coroutines or {\it fibers}, share a kernel thread, but each execute on their own stack, enabling a thread to do useful work for one fiber while another waits for a response from a trustee. This includes executing the local trustee fiber to service any incoming requests.

In this default setting, the synchronous {\tt apply()} function suspends the current fiber when it issues a request, scheduling the next fiber from the local ready queue to run instead. The local fiber scheduler will periodically poll for responses to outstanding requests, and resume suspended fibers as their blocking requests complete.

\subsection{Delegated context}
%In principle, much of this could be implemented with kernel threads, but in practice, a light-weight user threading implementation is both more efficient, and more practical as it allows us to run our own scheduler. 
For the purpose of future discussion, we define the term {\it delegated context} to mean the context where a delegated closure executes. Generally speaking, closures execute as part of a trustee fiber, on the trustee's stack. 
Importantly, blocking delegation calls are not permitted from within delegated context, and will result in a runtime assertion failure. 
In Section~\ref{s:nesting}, we describe multiple ways around this constraint.
\section{Core API}
\label{s:nesting}
The \trust{} API supports a variety of ways to delegate work, some of which we elide due to space constraints. Below, we describe the core functions in detail.

\subsection{\texttt{apply()}: synchronous delegation}
\begin{lstlisting}[numbers=none]
apply(c: FnOnce(&mut T)->U)->U
\end{lstlisting}

{\tt apply()} is the primary function for blocking, synchronous delegation as described in earlier sections. 
It takes a closure of the form {\tt |\&mut T| \{\}}, where {\tt T} is the type of the property. 
If the closure has a return value, apply returns this value to the caller. 

Importantly, \texttt{apply()} is synchronous, suspending the current fiber until the operation has completed.
Often, the best performance with {\tt apply()} is achieved when running multiple application fibers per thread. 
Then, while one fiber is waiting for its response, another may productively use the CPU.

\pagebreak

\subsection{\texttt{apply\_then()}: non-blocking delegation}
\begin{lstlisting}[numbers=none]
  apply_then(c: FnOnce(&mut T)->U, 
          then: FnOnce(U))
\end{lstlisting}
  

\begin{figure*}[ht]
  \centering
    \begin{lstlisting}[
		basicstyle=\small, %or \small or \footnotesize etc.
	]
let ct = trustee.entrust(17);           // create trust for shared counter set to 17
ct.apply_then(|c| { *c+=1; *c },         // increment counter and return its value
              |val| assert!(val==18));  // check return value once received
\end{lstlisting}
\caption{Asynchronous version of the example in \ref{f:minimal}.
The second closure runs on the client, once the result of the first closure is received from the trustee. }
\label{f:minimal_async}
\end{figure*}

Frequently, asynchronous (or non-blocking) application logic can allow the programmer to express additional concurrency either without running multiple fibers, or in combination with multiple fibers. 
Here, {\tt apply\_then()} returns to the caller without blocking, and does not produce a return value.
Instead, the second closure, {\tt then}, is called with the return value from the delegated closure, once it has been received. 
\ref{f:minimal_async} demonstrates the use of {\tt apply\_then()} following the pattern of \ref{f:minimal}.

The {\tt then}-closure is a very powerful abstraction, as it too is able to capture variables from the local environment, 
allowing it to perform tasks like adding the return value (once available) to a vector accessible to the caller. 
Here, Rust's strict lifetime rules automatically catch otherwise easily introduced use-after-free and dangling pointer problems, forcing
the programmer to appropriately manage object lifetime either through scoping or reference counted heap storage. 

%Similar to {\tt apply()}, the request is appended to the local pending request queue, and transmitted once a request line becomes available. 
Importantly, as {\tt apply\_then()} does not suspend the caller, it may freely be called from within delegated context. 

\subsection{\texttt{launch()}: apply in a trustee-side fiber}
\begin{lstlisting}[numbers=none]
  launch(c: FnOnce(&mut T)->U)->U
  launch_then(c: FnOnce(&mut T)->U, 
           then: FnOnce(U))

\end{lstlisting}

The most significant constraint imposed by \trust{} on the closure passed to {\tt apply()} and {\tt apply\_then()} is that the closure itself may not block. 
Blocking in delegated context means putting the trustee itself to sleep, preventing it from serving other requests, potentially resulting in deadlock.
In previous work \cite{rcl}, this problem was addressed by maintaining multiple server OS threads, and automatically switching to the next server when one server thread blocks. This avoids blocking the trustee, but imposes high overhead, resulting in considerably lower performance, as demonstrated in \cite{ffwd}.

In \trust{}, blocking in delegated context is prohibited: attempted suspensions in delegated context are detected at runtime, resulting in an assertion failure. Closures may still use {\tt apply\_then()}, but not the blocking {\tt apply()}. 
\footnote{Other forms of blocking, such as I/O waits or scheduler preemption, do not result in assertion failures. However, these can significantly impact performance if common, as blocking the trustee can prevent other threads from making progress.}

The lack of {\it nested blocking delegation} can be a significant constraint on the developer, and perhaps the most important limitation of \trust{}. 
Specifically, it affects modularity, as a library function that blocks internally, even on delegation calls, cannot be used from within delegated context.  

\begin{figure*}
  \centering\
  \includegraphics[height=1.8in]{figures/launch}
  \caption{Operation of {\tt launch()} vs {\tt apply()}. {\tt launch()} supports blocking calls, including nested delegation calls in the
  delegated closure, but incurs a higher minimum overhead. Solid arrows indicate requests, dotted arrows are delegation responses.}
  \label{f:launch}
\end{figure*}

To address this, without sacrificing the performance of the more common case, we provide a convenience function: {\tt launch()}, which offers all the same functionality as {\tt apply()}, but without the blocking restriction.
\ref{f:launch} describes {\tt launch()} from an implementation standpoint. {\tt launch()} creates a temporary fiber on the trustee's thread, which runs the closure. If this fiber is suspended, the client is notified, and the trustee continues to serve the next request. 
Once the temporary fiber resumes and completes execution of the closure, it then delivers the return value and resumes the client fiber via a second delegation call. Thus, if a delegated closure fails the runtime check for blocking calls, the developer can fix this by replacing the {\tt apply()} call, with a {\tt launch()} call. 

% The Rust type system does not let us statically detect whether a closure needs to be delegated with {\tt launch()}. Detecting it at runtime is trivial. 
% Automatically selecting {\tt apply()} or {\tt launch()} at runtime is feasible, but not without imposing significant overhead on the fast path.
% In general, because {\tt launch()} incurs higher overhead than {\tt apply()} (see \S\ref{s:eval_launch} for details), it is better used only as needed.
% That said, a developer who is eager to avoid any chance of a runtime assertion failure can simply substitute every {\tt apply()} call in their code with a {\tt launch()} call.

\subsubsection{Atomicity and \texttt{launch()}}
That said, a complicating factor with blocking closures executed by {\tt launch()} is that without further protection, property accesses are no longer guaranteed to be atomic:
while the newly created fiber is suspended, another delegation request may be applied to the property, resulting in a race condition. 
To avoid this risk, {\tt launch()} is implemented only for {\tt Trust<Latch<T>{>}}.
{\tt Latch<T>} is a wrapper type which provides mutual exclusion, analogous to {\tt Mutex<T>} except that it uses no atomic instructions, and thus may only be accessed by the fibers of a single thread.\footnote{In Rust terms, {\tt Latch<T>} does not implement {\tt Sync}.}
%More on nested blocking delegation requests in \S\ref{s:nested}.

\subsubsection{Leveraging Rust for safe and efficient delegation}
\label{s:noref}

Using the Rust type system, we ensure that delegated closures in \trust{} cannot capture values that contain any references or pointers. 

In principle, this is far stricter than what is necessary: the existing and pervasive Rust traits {\tt Send} and {\tt Sync} already describe
the types that may be safely moved and shared between threads, and this continues to hold within \trust{}. 

That said, safety does not guarantee performance. A common performance pitfall when writing delegation-based software is 
memory stalls on the trustee, which affects trustees disproportionately due to the polling nature of the delegation channel (see Section \ref{s:delegating}). Frequent cache misses and use of atomic instructions in delegated closures can substantially degrade trustee throughput vs. running closures with good memory locality. 

Generally speaking, cache line contention and use of atomic instructions are a natural result of sharing memory between threads. By prohibiting the capture of references and pointers, \trust{} makes accidental shared memory patterns of programming much less likely in delegated code, and encourage cache friendly pass-by-value practices. 

\subsubsection{Variable-size and other heap-allocated values}

Rust closures very efficiently and conveniently capture their environment, which {\tt apply()} sends whole-sale to the trustee. 
However, only types with a size known at compile time may be captured in a Rust closure (or even allocated on the stack). 

In conventional Rust code, variable size types, including strings, are stored on the heap, and referenced by a {\tt Box$<$T$>$} smart pointer. 
For the reasons described above (see Section \ref{s:noref}), we do not allow {\tt Box$<$T$>$} or other types that include pointers or references to be captured in a closure:
only pure values may pass through the delegation channel. 

As a result, variable size objects and other heap-allocated objects must be passed as explicit arguments rather than captured, so that they may be serialized before transmission over the delegation channel. 
For example, a {\tt Box$<$[u8]$>$} (a reference to a heap-allocated variable-sized array of bytes) cannot traverse the delegation channel. Instead, we encode a copy of the variable number of bytes in question into the channel, and pass this value to the closure when it is executed by the trustee. 
In practice, this takes the form of a slightly different function signature.

\begin{lstlisting}[numbers=none]
apply_with(c: FnOnce(&mut T, V)->U, w: V)->U
\end{lstlisting}

Here, the {\tt w:} argument is any type {\tt V:Serialize+Deserialize}, using the popular traits from the {\tt serde} crate. That is, any type that can be serialized and deserialized, may pass over the delegation channel in serialized form.
If more than one argument is needed, these may be passed as a tuple. Thus, to insert a variable-size key and value into an entrusted table, we might use:

\begin{lstlisting}[numbers=none]
  table_trust.apply_with(|table, (key, value)| 
    table.insert(key,value),(key,value))
\end{lstlisting}

We use the efficient {\tt bincode} crate internally for serialization.
As a result, while passing heap-allocated values does incur some additional syntax, the impact in terms of performance is minimal.
\section{Key Design and Implementation Details}

In this section, we delve deeper into the design and implementation of \trust{}, from the mechanics of delegating Rust closures and handling requests and responses, to asynchronous versions of {\tt apply()}.

\subsection{Delegating Closures}
\label{s:delegating}

The key operation supported by \trust{} is {\tt apply()}, which applies a Rust {\it closure} to the property referenced by the trust. A Rust closure consists of an anonymous function and a captured environment, which together is represented as a 128-bit {\it fat pointer}. Thus, to delegate a closure, a request must at minimum contain this fat pointer, and a reference to the property in question. 

One or more requests are written to the client's dedicated, fixed-sized {\it request slot} for the appropriate trustee.
That is, only the client thread may write to the request slot. 
For efficiency, if the captured environment of the closure fits in the request slot, we copy the environment directly to the slot, and update the fat pointer to reflect this change. 
A flag in the request slot indicates that new requests are ready to be processed. See \S\ref{st:slots} for details on request and response slot structure.

Responses are transmitted in a matching dedicated response slot. Leveraging the Rust type system, we restrict both requests and responses to types that can be serialized. 
The subtle implication of this is that the return value may not pass any references or pointers to trustee-managed data.\footnote{That said, we cannot prevent  determined Rust programmers from using {\tt unsafe} code to circumvent this restriction.}
While small closures with simple, known-and-fixed-size return types will generally yield the best performance, there is no limit beyond the serializability requirement on the size or complexity of closures and return types. %API does support complex compound return types as well, such as a \Verb"HashMap<String,Option<Vec<usize>>>". 

\subsection{Scheduling Delegation Work}

Generally speaking, a call to {\tt apply()} appends a request to a pending request queue, local to the requesting thread.
In the case of {\tt apply()}, the calling fiber is then suspended, to be woken up when the response is ready.
Pending requests are sent during response polling, and as soon as an appropriate request slot is available. 
The intervening time is spent running other fibers, including the local trustee fiber, and polling for responses/transmitting requests.

There is a throughput/latency trade-off between running application fibers, and polling for requests/responses: poll too often, 
and few requests/responses will be ready, wasting polling effort. 
Poll too seldom, and many requests/responses will have been ready for a long time, increasing latency.
Automatically tuning this trade-off is an area of ongoing research. 
That said, the current implementation performs delegation tasks in a fiber that is scheduled in FIFO order just as other fibers.
After serving incoming requests, this fiber polls for incoming responses and issues any enqueued outgoing requests as applicable.

\subsubsection{Local Trustee Shortcut}

When a Trust has the current thread as its trustee, it is superfluous to use delegation to apply the closure. Instead, it is just as safe, and more efficient, to simply apply the closure directly, since we know that no other closures will run until the provided closure has run to completion. As a reminder, we know this because delegated closures may not suspend the current fiber.

\subsection{Request and Response Slot Structure}
\label{st:slots}

\begin{figure*}[t]
  \centering
  \includegraphics[width=5in]{figures/request_slot}
  \caption{The fixed-size \trust{} request slot consists of a {\tt ready} bit, a request counter, and a variable number of variable-sized requests. 
  The response slot contains a matching {\tt bit}, as well as one (fixed, variable, or zero-sized) response per request in the matching request slot. 
  There is one dedicated pair of request/response slots for each trustee/client pair.
  }
  \label{f:requestslot}
\end{figure*}
Figure \ref{f:requestslot} illustrates the internal structure of the basic request and response slot design. 
A header consisting of a {\tt ready} bit and a request count, is followed by a variable number of variable-sized requests.
The value of the {\tt ready} bit is used to indicate whether a new request or set of requests has been written to the slot: 
if the bit differs from the {\tt ready} bit in the corresponding response slot, then a new set of requests is ready to be processed.

By default, the slot size is 1152 bytes, and the client may submit as many closures as it can fit within the slot. Here, the minimum size of a request is 24 bytes: a 128-bit fat pointer for the closure, and a regular 64-bit pointer for the property.
The captured environment of Rust closures have a known, fixed size, which is found in the vtable of the closure. 
For typical small captured environments, this is copied into the request slot, and the pointer updated to point at the new location. 
Serialized closure arguments are appendend next, followed by the next request. 

Responses are handled in a similar fashion, though there is no minimum response size. 
Responses are sent simultaneously for all the requests in the request slot. 
The size of each response is often statically known, in which case it is not encoded in the channel.
Any variable-size responses are preceded by their size. 

The size of each request is always known, either statically or at the time of submission, which means we can restrict the number of requests sent to what can be accommodated by the request slot. 
The size of the response is not always known at the time the request is sent.
In cases where the combined size of return values exceeds the space in the response slot, the trustee dynamically allocates additional memory to fit the full set of responses,
at a small performance penalty. 

\subsubsection{Two-part slot optimization}

In order to accommodate a broad range of application characteristics, including those with a single trustee and many clients, as well as a single client with many trustees, we introduce a small optimization beyond the basic design above. 
Rather than represent the request and response slots as monolithic blocks of bytes, we represent each as two blocks: a 128-byte primary block, and a 1024-byte overflow block; each request and response is written, in its entirety, to one or the other block. 

This addresses an otherwise problematic trade-off with respect to the request and response slot sizes: 
with a monolithic request slot of, say, one kilobyte, the trustee would be periodically scanning flags 1024 bytes apart, a very poor choice from a cache utilization perspective, unless the slots are heavily utilized.
A two-part design accommodates a large number of requests (where needed), but improves the efficiency of less heavily utilized request slots by spacing ready flags, and a small number of compact requests, more closely using a smaller primary request block.

\subsection{Reference counting for Trusts}
Trusts in \trust{} act as rust smart pointers that own the property, since they can be used to access and 
modify the property behind them. This means that we need to make sure that when all of the trusts are dropped, 
the property is also dropped and the associated memory is freed so as not to have memory leakage. To acheive 
that trusts need to be reference counted, but a naive integer count will not suffice due to a combination of 
the following two issues.

\begin{itemize}
	\item \trust{} does not support a blocking delegation operation when in the middle of another delegation 
	request. For example calling \textit{apply} on a trust from within a closure that itself is used for 
	delegation will cause the system to hang and never finish.
	\item If the integer used for counting references, let's call it refcount, is incremented and decremented 
	asynchronously, i.e. with a nonblocking delegation call, it could lead to use after free bugs. Let's 
	consider the following scenario: Thread A clones a trust with only one reference and sends an async request 
	to increment its refcount. The cloned trust is then sent to Thread B that drops it, sending another async 
	request to decrement its refcount. While the requests issued by the same fiber are guaranteed to be 
	completed in the same order they are issued, there is no such guaranty across multiple threads or even 
	fibers. It is entirely possible that the decrement request is processed first, making the refcount zero, 
	which results in the property being dropped. Thread A however still has a valid trust to this property 
	which it can use, expecting the property to still be available.
\end{itemize}

Not being able to use async delegation to increment and decrement the refcount leads to not being able to clone 
a trust from within a delegated context, which can quite restrictive. To get around that, \trust{} uses a new 
way to keep track of how many trusts are active at any time. Each trust is given a unique id at the time of 
creation, be it a new trust or a clone of an existing one. Instead of just using an integer, \trust{} uses a 
set of these ids associated with each property. Instead of incrementing or decrementing the refcount, clone 
and drop both issue a delegation request involving a symmetric difference operation. Symmetric difference 
is defined as adding an element to a set if it is not already a member, and removing it from the set if it is. 
This way, regardless of the order in which requests originating in clone and drop are processed, the first one 
will add the trust id to the set and later one will remove it, allowing us to use async delegation for both. 
This, in turn, enables us to clone a trust within a delegated context.

\section{Evaluation}
\label{s:eval} 

\begin{figure*}[ht]
  \centering
  \includegraphics[width=0.8\textwidth]{figures/plots/throughput_manyvars}
  \caption{Fetch-and-add throughput vs. object count. Uniform access distribution. }
  \label{f:fna_variables}
\end{figure*}
\begin{figure*}[ht]
  \centering
  \includegraphics[width=0.8\textwidth]{figures/plots/throughput_manyvars_zipf}
  \caption{Fetch-and-add throughput vs. object count. Zipfian access distribution, $\alpha=1$.}
  \label{f:fna_variables_zipf}
\end{figure*}



Below, we evaluate the performance of \trust{} in two ways: 1) on both microbenchmarks, designed to stress test the core mechanisms behind \trust{} and locking, and 2) on end-to-end application benchmarks,
which measure the performance impact of \trust{} in the context of a complete system and a more realistic use case. 

\subsection{Fetch and Add: Throughput}

For our first microbenchmark, we use a basic fetch-and-add application. Here, a number of threads repeatedly increment a counter chosen from a set of one or more, and fetches the value of the counter. In common with prior work on synchronization and delegation \cite{dice2011flath,calciu2013delegation,petrovic2015delegation,fatourou2011combining,hendler2010flat,oyama1999combining,yew1987combining,shalev2006combining,david2013everything}, we also include a single {\tt pause} instruction in both the critical section and the delegated closures. The counter is chosen at random, either from a uniform distribution, or a zipfian distribution. Each thread completes 1 million such increments. 
In this section, each data point is the result of a single run.

Below, we primarily evaluate on a two-socket Intel Xeon CPU Max 9462, of the Sapphire Rapids architecture. This machine has a total of 64 cores, 128 hyperthreads, and 384 GB of RAM. Unless otherwise noted, we use 128 OS threads. In testing, several older x86-64 ISA processors have shown similar trends -- these results are not shown here.
 For locking solutions, we use standard Rust {\tt Mutex<T>} and the spinlock variant provided by the Rust {\tt spin-rs-0.9.8} crate, as well as MCSLock$<$T$>$ provided by the Rust {\tt synctools-0.3.2} crate. For {\tt Trust<T>}, we show results for blocking delegation (Trust) as well as nonblocking delegation (Async).
In \ref{f:fna_variables}, we also include TCLocks, a recent combining approach offering a transparent replacement for standard locks, via the Litl lock wrapper \cite{litl} for {\tt pthread\_mutex}. 
To be able to evaluate this lock, we wrote a separate C microbenchmark, matching the Rust version. In the interest of an apples-to-apples comparison, we first verified that the reported performance with stock {\tt pthread\_mutex} on the C microbenchmark matched the Rust {\tt Mutex<T>} performance in our Rust microbenchmark.

Below, the {\tt Trust} results may be seen to represent any application with ample concurrency available in the form of conventional synchronous threads. {\tt Async} represents applications where a single thread may issue multiple simultaneously outstanding requests, e.g. a key-value store or web application server. 
Applications with limited concurrency are not well suited to delegation, except where the delegated work is itself substantial, which is not the case for this fetch-and-add benchmark. 
We further report results with both letting all cores serve as both clients and trustees ($shared$), as well as with an ideal number of cores dedicated serve only as trustees ($dedicated$).

\subsubsection{Uniform Access Pattern}


\ref{f:fna_variables} illustrates the performance of several solutions on the uniform distribution version of this benchmark. 
For a very small number of objects, no data points are reported for some of the lock types - this is because the experiment took far too long to run due to severe congestion collapse. 

{\tt Trust<T>} substantially outperforms locking under congested conditions.
Between 1--16 objects, the performance advantage is 8--22$\times$ the best-performing MCSLock. 
%Where each OS thread hosts only a single application fiber, the advantage is less pronounced - 2--3$\times$, and only for 4 variables or less. 
%While queue-based locks have been shown to better avoid severe congestion collapse, it was shown in FFWD \cite{ffwd} that these locks still perform poorly, around 3 MOps, compared to delegation, under congested conditions. 
For larger numbers of objects, the overhead of switching between fibers becomes apparent, as asynchronous delegation is able to reach a higher peak performance.  
In entirely uncongested settings, with 10$\times$ as many objects as there are threads, locking is able to match asynchronous delegation performance.
TCLocks \cite{tclocks} was the only lock type to complete the single-lock experiment within a reasonable time. 
It consistently outperforms spinlocks under congestion, and remains competitive with Mutex and MCS on highly congested locks.
However, TCLocks appear to trade their transparency for high memory and communication overhead, making it unable to compete performance-wise beyond highly congested settings. 
\footnote{TCLocks performance appears somewhat architecture dependent. In separate runs on our smaller Skylake machines, TCLocks were able to outperform Mutex by $\approx$50\% under the most extreme contention (a single lock).} 
Moreover, we struggled to apply TCLocks to memcached (which consistently crashed under high load), as well as to Rust programs (as Rust now uses built-in locks rather than {\tt libpthreads} wrappers). We thus elide TCLocks from the remainder of the evaluation.

%For \trust{}, performance increases with the number of objects until it reaches the number of trustees. After that, there is not much benefit to having more objects, except for distributing work evenly between trustees. 

\begin{figure*}[ht]
  \centering
  \includegraphics[width=0.8\textwidth]{figures/plots/latency_vs_load}
  \caption{Mean latency vs. offered load. Uniform access distribution.}
  \label{f:latency_vs_load}
\end{figure*}

\begin{figure*}[ht]
  \centering
  \includegraphics[width=0.8\textwidth]{figures/plots/latency_vs_load_zipf}
  \caption{Mean latency vs. offered load. Zipfian access distribution, $\alpha=1$.}
  \label{f:latency_vs_load_zipf}
\end{figure*}

\subsubsection{Skewed Access Pattern: Zipfian distribution}

Zipf's law \cite{zipf1949} elegantly captures the distribution of words in written language. In brief, it says that the probability of word occurrence $p_w$ is distributed according to the rank $r_w$ of the word, thus: $p_w \propto {r_w}^{-\alpha}$, where $\alpha \sim 1$.
Similar relationships, often called ``power laws'', are common in areas beyond written language \cite{zipf1949,boltzmann,Ojovan_2006,DODDS20019,Venture}, sometimes with a greater value for $\alpha$. The higher the $\alpha$, the more pronounced is the effect of popular keys, resulting in congestion. 

\ref{f:fna_variables_zipf} shows the results of our fetch-and-add experiment, but with objects selected according to a zipfian distribution ($\alpha=1$) instead of the uniform distribution above, representing a common skewed access distribution. 

With this skewed access pattern, \trust{} overwhelmingly outperforms locking across the range of table sizes. 
This is explained by the relatively low throughput of a single lock. 
In our experiments, even MCSLocks, known for their scalability, offer at best 2.5 MOPs. 
When a skewed access pattern concentrates accesses to a smaller number of such locks, low performance is inevitable.
By comparison, a single \trust{} trustee will reliably offer 25 MOPs, for similarly short critical sections.
For more highly skewed patterns, where $\alpha > 1$ (not shown), the curve grows ever closer to the horizontal as performance is bottlenecked by a small handful of popular items.


\subsection{Fetch and Add: Latency}

Next we measure mean latency for a scenario with 64 objects (uniform access distribution), and 1,000,000 objects (Zipfian access distribution), while varying the offered load. 
We show delegation results with 8 dedicated trustee cores, and with 64 shared trustee cores\footnote{The evaluation system has 64 cores, 128 hardware threads. In the vast majority of cases, having both hardware threads of each core work as trustees results in reduced performance.}.
We also plot the results for a spinlock, a standard Rust mutex, and an MCS lock as above. 

At low load, low contention results in low latency for locking, an ideal situation for locks.
%For low load, there is little lock contention, which results in low latency for locking. 
%This latency is representative of an ideal use case for locks. 
However, as load increases, the locks eventually reach capacity, resulting in a rapid rise in latency. 
With \trust{}, even low load incurs significant latency, due to message passing overhead. 
However, due to the much higher per-object capacity available, latency increases slowly with load until the capacity is reached. 
Thus, \trust{} offers stable performance over a wide range of loads, at the cost of increased latency at low load. 
The higher latency does mean that to take full advantage of delegation, applications need to have 
ample parallelism available.%, such that clients typically have other useful work to do, such as more requests to prepare, while they await the next trustee response. 

%It's worth noting that with the Zipfian access distribution, even with 1,000,000 objects, no lock type shows a significant latency advantage over delegation even at very low loads. 
%This is because Zipfian access patterns tend to concentrate much of the load to a small number
%of keys, here the low capacity of each individual lock is quickly exhausted, resulting in congestion collapse.

For both Uniform and Zipfian access distributions, we also measured 99.9th percentile (tail) latency (not shown). Overall, tail latency with locking (all types) tended to be approximately 10$\times$ the mean latency, in low-congestion settings. Delegation tail latency with a dedicated trustee, meanwhile, was 2.5$\times$ the mean, making delegation tail latency under low load only 2--3$\times$ that of locking.% Under load approaching capacity, both latency and tail latency grow rapidly to extremely high levels, for both locking and delegation. In this regime, 
%the figure of merit is capacity, rather than latency.

It's also worth noting the difference between 8 dedicated trustees, and 64 trustees on threads shared with clients. The latency when sharing the thread with clients is naturally higher than when using trustees dedicated to trustee work. However, as load increases having more trustees available to share the load results in better performance. Using all the cores for trustees all the time also eliminates an important tuning knob in the system. 



% \subsection{Nested delegation / locking}

% We now evaluate the performance of \trust{} in a nested delegation setting. In this microbenchmark, we atomically modify two objects together, where the two objects may be hosted by different trustees, as described in \ref{s:nesting}. 

% In Figures \ref{f:nesting} (uniform) and \ref{f:nesting_zipf} (Zipfian), we vary the number of variables, and measure the total throughput available. 
% Similar to the non-nested case, delegation fares significantly better than locking for popular objects, whether due to the access distribution or the total number of objects. Also, locking does considerably better than delegation where there is no congestion. 
% This is primarily for the same reason as in the non-nested case. 
% However, another contributing factor is the added coordination required in the nested delegation case: instead of doubling the number of messages, nesting adds a third message, so the maximum theoretical throughput for a two-object transaction is one third of the throughput of a single-object transaction. 


\subsection{Concurrent key-value store}
\label{s:kvstore}

\begin{figure*}[ht]
  \centering
  \includegraphics[width=0.8\textwidth]{figures/plots/var-keyspace-uniform.pdf}
  \caption{Key-value store throughput, with 5\% writes and varying table size. Uniform access distribution}
  \label{f:kvstore_tput}
\end{figure*}

\begin{figure*}[ht]
  \centering
  \includegraphics[width=0.8\textwidth]{figures/plots/var-keyspace-zipfian.pdf}
  \caption{Key-value store throughput, with 5\% writes and varying table size. Zipfian access distribution}
  \label{f:kvstore_tput_zipf}
\end{figure*}

\begin{figure*}[ht]
  \centering
  \includegraphics[width=0.8\textwidth]{figures/plots/var-write-uniform.pdf}
  \caption{Key-value store throughput, with varying write percentage. Uniform access distribution}
  \label{f:kvstore_tput_writes}
\end{figure*}

\begin{figure*}[ht]
  \centering
  \includegraphics[width=0.8\textwidth]{figures/plots/var-write-zipfian.pdf}
  \caption{Key-value store throughput, with varying write percentage. Zipfian access distribution}
  \label{f:kvstore_tput_writes_zipf}
\end{figure*}

For a more complete end-to-end evaluation, we implement a simple TCP-based key-value store, backed by a concurrent dictionary.
Here, we run a multi-threaded TCP client on one machine, and our key-value store TCP server on another, identical machine.
The two machines are connected by 100 Gbps Ethernet.
We compare our \trust{} based solution to Dashmap \cite{dashmap}, one of the highest-performing concurrent hashmaps available as a public Rust crate,
as well as to our own na\"ively sharded Hashmap, using Mutex or Readers-writer locks and the Rust {\tt std::collections::HashMap$<$K, V$>$}.
Dashmap is a heavily optimized and
well-respected hash table implementation, which is regularly benchmarked against competing designs.

We implement the key-value store as a multi-threaded
server, where each worker-thread receives \texttt{GET} or \texttt{PUT} queries from one or more connections,
and applies these to the back-end hashmap. Both reading requests and sending results is done in batches, so as to
minimize system call overhead. Moreover, the client accepts responses out-of order, to minimize waiting.
The TCP client continuously maintains a queue of parallel queries over the socket, such that the server always has
new requests to serve.
In the experiments, we dedicate one CPU core to each worker thread.

For our sharded hashmaps, we create a fixed set of 512 shards, using many more locks than threads to reduce lock contention.
Dashmap uses sharding and readers-writer locks internally, but exposes a highly efficient concurrent hashmap API.
For our \trust{} based key-value store, we use 16 and 24 cores to run trustees (each hosting a shard of the table) exclusively,
and the remaining cores for socket workers.
They are named Trust16 and Trust24, respectively.
Socket workers delegate all hash table accesses to trustees.
The key size is 8 bytes and the value size is 16 bytes in the experiments.
Prior to each run, we pre-fill the table, and report results from an average of 10 runs.
%each lasting approximately \je{?} seconds.

\ref{f:kvstore_tput} and \ref{f:kvstore_tput_zipf} show the results from this small key-value store application,
for a varying total number of keys with 5\% write requests and 95\% read request,
and Uniform as well as Zipfian \cite{zipf1949} access distributions.
For Zipfian access, we use the conventional $\alpha=1$.
Overall, similar to the microbenchmark results, we find that the delegation-based solution
performs significantly better when contention for keys is high. However, due to the considerably higher complexity of
this application, the absolute numbers are lower than in our microbenchmarks. The relative advantage for delegation is also somewhat smaller,
as some parts of the work of a TCP-based key-value store are already naturally parallel.

For the Uniform distribution and 5\% writes, all the solutions perform similarly above 1,000 keys,
a large enough number that there is no significant contention. With 100 keys and less, \trust{} enjoys a large advantage even under uniform access distribution.
With a Zipfian access distribution, accesses are concentrated at the higher-ranked keys, leading to congestion.
In this setting, \trust{} trounces the competition, offering substantially higher performance across the full 1--100,000,000 key range.
It is interesting to note, also, that the Zipfian access distribution is where the carefully optimized design of
Dashmap shines, while it offers a fairly limited advantage over a na\"ive sharded design with readers-writer locks
on uniform access distributions. This speaks to the importance of efficient critical sections in the presence of lock congestion.

The throughput of Trust16 is higher than Trust24 with 1,000--100,000 keys because it is of low cost to manage a relatively
small key space, while Trust16 can dedicate more resources to handle socket connections.
However, the performance of Trust16 starts to degrade with more keys, because the limited number of trustees fall
short when managing larger key spaces. With 24 trustees, the performance can be maintained at a high level.
The difference between Trust16 and Trust24 suggests an important direction of future research: For I/O heavy processes
like key-value stores, dedicated trustees will often outperform sharing the core between trustees and clients.
However, it is non trivial to correctly choose the number of trustees.
Automatically adjusting the number of cores dedicated to trustee work at runtime would be preferable.

In principle, readers-writer locks have a major advantage over \trust{} in that they allow concurrent reader access,
while \trust{} exclusively allows trustees to access the underlying data structure.
To better understand this dynamic, \ref{f:kvstore_tput_writes} and \ref{f:kvstore_tput_writes_zipf}
show key-value store throughput over a varying percentage of writes.

Here, we use 1,000 keys for the Uniform access distribution, and 10,000,000 keys for Zipfian access distribution.
We note that these are table sizes where lock-based approaches hold an advantage in \ref{f:kvstore_tput} and \ref{f:kvstore_tput_zipf}. 
For Uniform access patterns, where there is limited contention given the table size of 1,000 keys, the
impact of the write percentage is muted. For lock-based designs, the performance does drop somewhat, but remains
at a high level even with 100\% writes. 

It is interesting to note that \trust{} performance increases modestly with the write percentage. One reason behind this is that in our 
key-value store, the closures issued by reads by necessity have large return values, while the closures issued by writes have no return values at all. This may allow the trustee to use only the first, small part of the return slot, occasionally saving two LLC cache misses per round-trip.

% We hypothesize that this is due to differences in memory allocation: with writes, there is typically the need to allocate heap memory for a new item, and release the memory for an item being replaced. In a lock-based design, this memory may be allocated by one core, then released by another, leading to contention for cache lines and potentially locks within the heap allocator. 
% With \trust{}, memory is freed by the same core that allocated it, improving memory locality. 

With the Zipfian access distribution, even with 10,000,000 keys, contention remains a bigger concern, especially for Mutex. All four designs exhibit reduced performance with increased write percentages, but again, \trust{} 
proves more resilient. 
The efficiency advantage of Dashmap over our na\"ive lock-based designs is on full display with the Zipfian access 
distribution and a high write percentage. 
That said, the fundamental advantage of \trust{} over locking in this application is clear.

\section{Legacy Application: Memcached}
\label{s:memcached}

We also port memcached version 1.6.20 to \trust{} to demonstrate both the applicability and performance impact on legacy C applications.
Memcached is a multi-threaded key-value store application. Its primary purpose is serving PUT and GET requests with string keys and values over standard TCP/IP sockets. Internally, memcached contains a hash-table type data structure with external linkage and fine-grained per-item locking. By default, memcached is configured to use a fixed number of worker threads. Incoming connections are distributed among these worker threads. Each worker thread uses the {\tt epoll()} system call to listen for activity on all its assigned connection. Each connection to a memcached server traverses a fairly sophisticated state machine, a pipelined design that is aimed at maximizing performance when each thread serves many concurrent connections with diverse behaviors. The state machine will process requests in this sequence: receive available incoming bytes, parse one request, process the request, enqueue the result for transmission, and transmit one or more  results. 

For our port to \trust{}, we eliminate the use of most locks, and instead divide the internal hash table and supporting data structures into one or more shards, and delegate each shard to one of potentially multiple trustees. 
Thus, instead of acquiring a lock, we delegate the critical section to the appropriate trustee for the requested operation.  
Our ported version follows the original state machine design, with one key difference:
for each incoming request on the socket, we make an asynchronous delegation request using {\tt apply\_then}, then 
move on to the next request without waiting for the response from the trustee.  
That is, rather than sequentially process each incoming request, we leverage asynchronous delegation to capture additional concurrency. 

A complicating factor in this asynchronous approach results from {\tt memcached} being initially designed for synchronous operation with locking. 
For any one trustee-client pair, even asynchronous delegation requests are executed in-order, and responses arrive in-order.
However, this is not guaranteed for requests issued to different trustees. 
Consequently, the memcached socket worker thread must order the responses before they are transmitted over the network socket to the remote client.
By contrast, our delegation-native key-value store in Section \ref{s:kvstore} sends responses out of order over the socket, and instead includes a request ID in the response.

Another difference worth mentioning is that we don't allow delegation clients (in this case, the memcached socket worker thread) to access delegated data structures at all. 
This means that instead of a pointer to a value in the table, clients receive a copy of the value. 
This significantly improves memory locality and simplifies memory management, since every value has a single owner. 
However, it does incur extra copying, which may reduce performance under some circumstances.\footnote{This can become a problem when values are large. For this use case, \trust{} includes an equivalent of Rust's {\tt Arc<T>} which allows multiple ownership of read-only values. %\je{make this true}
}

In practice, because memcached is written in C and \trust{} is written in Rust, we cannot directly add delegation to the memcached source code. We address this in a two-step process: first, for any task that requires delegation, we create a 
minimal Rust function that performs that specific task. That is, a custom Rust function that becomes part of the 
memcached code base. Typically, such a function locates the appropriate Trust or TrusteeReference, and delegates a single closure.
Second, we break out the critical sections in the C code into separate inner functions that may be called from Rust. Thus, to delegate a C critical section, we simply call the inner function from a delegated Rust closure. 

Our port of Memcached to \trust{} has approximately 600 lines of added, deleted or modified lines of code, out of 34,000+ lines total. 
This number includes approximately 200 of lines which were simply cut-and-pasted into the new inner functions for critical sections.
In addition, we introduced approximately 350 new lines of Rust code, to provide the interface between the C and Rust environments.

\subsection{Evaluation}
%\begin{figure*}
\begin{figure}[ht]
  \centering
  \includegraphics[width=0.8\textwidth]{figures/plots/size}
  \caption{Memcached throughput with varying table size. Uniform access distribution. S: stock memcached. }
  \label{f:memcached_tput}
\end{figure}

\begin{figure}[ht]
  \centering
  \includegraphics[width=0.8\textwidth]{figures/plots/size_zipf}
  \caption{Memcached throughput with varying table size. Zipfian access distribution.}
  % \caption{Zipfian access distribution, $\alpha=1$. S: stock memcached.}
  \label{f:memcached_tput_zipf}
\end{figure}

To understand the performance of our delegated Memcached, we use the memtier benchmark client (version 1.4.0) with our delegated Memcached as well as stock memcached. 
For the cleanest results, but without loss of generality, we configure memcached with a sufficiently large hash power and available memory to eliminate table resizing and evictions. 
We also limit our evaluation to the conventional memcached PUT/GET operations.
Recent versions of memcached feature an optional new cache eviction scheme, which trades less synchronization for the need for a separate maintenance thread. For stock memcached, we evaluated both the traditional eviction scheme and the new one. We show results for the new scheme, which scales much better for write-heavy workloads and is otherwise similar in our setting. For our ported version, we use the traditional eviction scheme, maintaining one LRU per shard. 
Eviction is not relevant here, as we provide ample memory relative to the table size. 

The server and client run on separate machines, connected by 100Gbps Mellanox-5 Ethernet interfaces via a 100Gbps switch. Both client and server machines are 28-core, two-socket systems with Intel {\it Sandy Bridge} CPUs and 256 GB of RAM. The machines run Ubuntu Linux with kernel version 5.15.0.
Unless otherwise noted, we structure the experiments as follows: start a fresh {\tt memcached} instance. Populate the table with the indicated number of key-value pairs, then run measurements with 1\% writes, 5\% writes, and 10\% writes. 
After this, we start over with a new, empty {\tt memcached} instance. Each data point represents a single experiment, each set to last 20 seconds. For each, unless otherwise noted, we choose {\tt memcached} and {\tt memtier} parameters to maximize throughput. 
By default, this means 28 {\tt memcached} threads pinned to hardware threads 0--27. Running with 56 hardware threads did not yield any further performance improvement.
On the memtier side, we configure 28 threads, with four clients per thread, and pipelining set to 48.

\ref{f:memcached_tput} and \ref{f:memcached_tput_zipf} illustrates the throughput of {\tt memcached} as we vary the number of keys in the table. 
While the absolute numbers are significantly lower than in the microbenchmarks and the key-value store, the overall picture from {\tt memcached} corresponds well with previous experiments. 

Using \trust{} results in performance improvements of more than 5$\times$ when accessing popular objects, 
whether this popularity is due to a uniform access distribution across a smaller number of keys, or a Zipfian distribution over millions of key-value pairs. 
When all items are accessed infrequently, locking suffers very little contention, and has the advantage of better distributing the work across cores. Here, this results in performance competitive with delegation, at least for read-heavy workloads.

The stock version is heavily affected by writes, due to the extra work required for these operations. This includes memory allocation, LRU updates as well as table writes, all of which involve synchronization in a lock-based design. With \trust{}, all such operations are local to the shard/trustee, and do not require synchronization.  
With 5\% of writes, stock memcached loses $\approx$40\% of its performance, while the \trust{} version sees only a minor performance penalty, resulting in delegation outperforming locking in this setting for the entire range of table sizes. While not shown, this trend continues with even more writes.%, yielding as much as \je{XXX} higher performance for 100\% writes. 
\section{Hybrid Delegation}

\begin{figure*}[]
    \centering
    \begin{subfigure}[]{0.8\textwidth}
        \centering
        \includegraphics[width=\columnwidth]{figures/plots/motivation-zipfian-95r.pdf}
        \caption{Zipfian}
        \label{hybrid:fig:motivation-zipfian-95r}
    \end{subfigure}
    \centering
    \begin{subfigure}[]{0.8\textwidth}
        \centering
        \includegraphics[width=\columnwidth]{figures/plots/motivation-uniform-95r.pdf}
        \caption{Uniform random}
        \label{hybrid:fig:motivation-uniform-95r}
    \end{subfigure}
    \caption{The throughput of key-value stores using compared systems with different key space size (5\% write)}
    \label{hybrid:fig:motivation-95r}
\end{figure*}

\ref{f:kvstore_tput} shows that delegation's performance advantage depends on the workload and lock contention 
levels. This raises the question of whether it is possible to get the best of both worlds. In his PhD thesis, 
Dr. Chen \cite{chen_thesis} introduced HybridLock, a synchronization mechanism that dynamically switches between 
traditional locking and delegation based on workload conditions. When a HybridLock is created, it assumes low 
contention and starts with locking as default. If, during the runtime, it detects that the lock is highly contented, 
it switches to using delegation manage access to the associated property. By efficiently detecting contention and 
switching between the two modes, HybridLock can leverage delegation to achieve high performance in a highly 
contented system while keeping the overhead during low contention workload to a minimum. \ref{hybrid:fig:motivation-95r} 
is taken from Dr. Chen's thesis and shows that HybridLock can maintain high performance across contention levels.
\section{Conclusion}

Synchronizing access to shared data is essential for ensuring correct behavior in multithreaded applications. 
This paper introduces \trust{}, a delegation based synchronization mechanism for safe, highly performance 
concurrent access to shared mutable state in Rust. \trust{} provides evidence that delegation can be a competitive 
alternative to locking in real systems. The programming model integrates cleanly in Rust, making delegation an 
accessible option for developers. This work lays the groundwork for additional language and runtime support to 
unlock the performance and scalability benefits of delegation-based designs.
